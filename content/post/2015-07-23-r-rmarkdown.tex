% Options for packages loaded elsewhere
\PassOptionsToPackage{unicode}{hyperref}
\PassOptionsToPackage{hyphens}{url}
%
\documentclass[
]{article}
\usepackage{lmodern}
\usepackage{amssymb,amsmath}
\usepackage{ifxetex,ifluatex}
\ifnum 0\ifxetex 1\fi\ifluatex 1\fi=0 % if pdftex
  \usepackage[T1]{fontenc}
  \usepackage[utf8]{inputenc}
  \usepackage{textcomp} % provide euro and other symbols
\else % if luatex or xetex
  \usepackage{unicode-math}
  \defaultfontfeatures{Scale=MatchLowercase}
  \defaultfontfeatures[\rmfamily]{Ligatures=TeX,Scale=1}
\fi
% Use upquote if available, for straight quotes in verbatim environments
\IfFileExists{upquote.sty}{\usepackage{upquote}}{}
\IfFileExists{microtype.sty}{% use microtype if available
  \usepackage[]{microtype}
  \UseMicrotypeSet[protrusion]{basicmath} % disable protrusion for tt fonts
}{}
\makeatletter
\@ifundefined{KOMAClassName}{% if non-KOMA class
  \IfFileExists{parskip.sty}{%
    \usepackage{parskip}
  }{% else
    \setlength{\parindent}{0pt}
    \setlength{\parskip}{6pt plus 2pt minus 1pt}}
}{% if KOMA class
  \KOMAoptions{parskip=half}}
\makeatother
\usepackage{xcolor}
\IfFileExists{xurl.sty}{\usepackage{xurl}}{} % add URL line breaks if available
\IfFileExists{bookmark.sty}{\usepackage{bookmark}}{\usepackage{hyperref}}
\hypersetup{
  pdftitle={The rising blue tide.},
  pdfauthor={Jaffa Romain, Taojun Wang, and Anees Shaikh},
  hidelinks,
  pdfcreator={LaTeX via pandoc}}
\urlstyle{same} % disable monospaced font for URLs
\usepackage[margin=1in]{geometry}
\usepackage{color}
\usepackage{fancyvrb}
\newcommand{\VerbBar}{|}
\newcommand{\VERB}{\Verb[commandchars=\\\{\}]}
\DefineVerbatimEnvironment{Highlighting}{Verbatim}{commandchars=\\\{\}}
% Add ',fontsize=\small' for more characters per line
\usepackage{framed}
\definecolor{shadecolor}{RGB}{248,248,248}
\newenvironment{Shaded}{\begin{snugshade}}{\end{snugshade}}
\newcommand{\AlertTok}[1]{\textcolor[rgb]{0.94,0.16,0.16}{#1}}
\newcommand{\AnnotationTok}[1]{\textcolor[rgb]{0.56,0.35,0.01}{\textbf{\textit{#1}}}}
\newcommand{\AttributeTok}[1]{\textcolor[rgb]{0.77,0.63,0.00}{#1}}
\newcommand{\BaseNTok}[1]{\textcolor[rgb]{0.00,0.00,0.81}{#1}}
\newcommand{\BuiltInTok}[1]{#1}
\newcommand{\CharTok}[1]{\textcolor[rgb]{0.31,0.60,0.02}{#1}}
\newcommand{\CommentTok}[1]{\textcolor[rgb]{0.56,0.35,0.01}{\textit{#1}}}
\newcommand{\CommentVarTok}[1]{\textcolor[rgb]{0.56,0.35,0.01}{\textbf{\textit{#1}}}}
\newcommand{\ConstantTok}[1]{\textcolor[rgb]{0.00,0.00,0.00}{#1}}
\newcommand{\ControlFlowTok}[1]{\textcolor[rgb]{0.13,0.29,0.53}{\textbf{#1}}}
\newcommand{\DataTypeTok}[1]{\textcolor[rgb]{0.13,0.29,0.53}{#1}}
\newcommand{\DecValTok}[1]{\textcolor[rgb]{0.00,0.00,0.81}{#1}}
\newcommand{\DocumentationTok}[1]{\textcolor[rgb]{0.56,0.35,0.01}{\textbf{\textit{#1}}}}
\newcommand{\ErrorTok}[1]{\textcolor[rgb]{0.64,0.00,0.00}{\textbf{#1}}}
\newcommand{\ExtensionTok}[1]{#1}
\newcommand{\FloatTok}[1]{\textcolor[rgb]{0.00,0.00,0.81}{#1}}
\newcommand{\FunctionTok}[1]{\textcolor[rgb]{0.00,0.00,0.00}{#1}}
\newcommand{\ImportTok}[1]{#1}
\newcommand{\InformationTok}[1]{\textcolor[rgb]{0.56,0.35,0.01}{\textbf{\textit{#1}}}}
\newcommand{\KeywordTok}[1]{\textcolor[rgb]{0.13,0.29,0.53}{\textbf{#1}}}
\newcommand{\NormalTok}[1]{#1}
\newcommand{\OperatorTok}[1]{\textcolor[rgb]{0.81,0.36,0.00}{\textbf{#1}}}
\newcommand{\OtherTok}[1]{\textcolor[rgb]{0.56,0.35,0.01}{#1}}
\newcommand{\PreprocessorTok}[1]{\textcolor[rgb]{0.56,0.35,0.01}{\textit{#1}}}
\newcommand{\RegionMarkerTok}[1]{#1}
\newcommand{\SpecialCharTok}[1]{\textcolor[rgb]{0.00,0.00,0.00}{#1}}
\newcommand{\SpecialStringTok}[1]{\textcolor[rgb]{0.31,0.60,0.02}{#1}}
\newcommand{\StringTok}[1]{\textcolor[rgb]{0.31,0.60,0.02}{#1}}
\newcommand{\VariableTok}[1]{\textcolor[rgb]{0.00,0.00,0.00}{#1}}
\newcommand{\VerbatimStringTok}[1]{\textcolor[rgb]{0.31,0.60,0.02}{#1}}
\newcommand{\WarningTok}[1]{\textcolor[rgb]{0.56,0.35,0.01}{\textbf{\textit{#1}}}}
\usepackage{graphicx,grffile}
\makeatletter
\def\maxwidth{\ifdim\Gin@nat@width>\linewidth\linewidth\else\Gin@nat@width\fi}
\def\maxheight{\ifdim\Gin@nat@height>\textheight\textheight\else\Gin@nat@height\fi}
\makeatother
% Scale images if necessary, so that they will not overflow the page
% margins by default, and it is still possible to overwrite the defaults
% using explicit options in \includegraphics[width, height, ...]{}
\setkeys{Gin}{width=\maxwidth,height=\maxheight,keepaspectratio}
% Set default figure placement to htbp
\makeatletter
\def\fps@figure{htbp}
\makeatother
\setlength{\emergencystretch}{3em} % prevent overfull lines
\providecommand{\tightlist}{%
  \setlength{\itemsep}{0pt}\setlength{\parskip}{0pt}}
\setcounter{secnumdepth}{-\maxdimen} % remove section numbering

\title{The rising blue tide.}
\author{Jaffa Romain, Taojun Wang, and Anees Shaikh}
\date{2020-10-01T21:13:14-05:00}

\begin{document}
\maketitle

Date: 2020-10-06 Authored by: Jaffa Romain, Anees Shaikh, and Taojun
Wang.

\hypertarget{non-technical-executive-summary}{%
\section{Non-Technical executive
summary}\label{non-technical-executive-summary}}

For this month's polling update, we'd like to provide an update on the
Ontario riding of Richmond Hill to the Liberal Party of Canada. Richmond
Hill includes the following neighbourhoods:

\begin{itemize}
\tightlist
\item
  Elgin Mills
\item
  Bayview North
\item
  Bayview South
\item
  North Richvale
\item
  Hillsview
\item
  Bayview Hill
\item
  South Richvale
\item
  Langstaff
\item
  Doncrest
\end{itemize}

Since being represented in the House of Commons, the Liberal party has
dominated in federal elections, winning three consecutive elections
between 2004-2008.However, recent elections have not only shown a
decline in Liberal voters, but also an increase in Conservative votes.
Richmond hill is now a swing riding, where one party sweeping the riding
could be decisive in terms of winning the election. Swing ridings are
electoral districts that elect different parties during elections. This
should be an important area of focus for the Liberal party as it may
decide the fate of the next election. The party benefits by receiving
this poll in a few ways. Chiefly, this polling update aims to answer a
few questions and provide the party with actionable data. Ultimately,
there is an opportunity cost associated with every decision a political
party makes. By opting for a data driven approach, the Liberal party can
make a well informed decision to invest in an area or realise that they
might have to focus their efforts in ridings that may prove more
fruitful.

Some of the questions include:

\begin{itemize}
\item
  How have residents voted in the past? This helps us establish whether
  they've voted predominantly one party, or are a swing party. The
  implications of this help the Liberal party plan accordingly. In the
  event of it being a swing riding, the party might have to concentrate
  it's efforts to swing the vote their way.
\item
  What are their current attitudes towards the Liberal Party?
  Essentially, this questions provides the party with a pulse of how the
  riding is currently doing. The benefit of this question will help the
  party divert its efforts accordingly. If the riding holds a favourable
  view, then the party may better spend its efforts elsewhere.
\item
  How do they measure on some important demographic variables? This
  information helps the party understand their voter base better. By
  gaining this understanding, the Liberals can create an effective
  strategy to drive up the votes here and possibly apply this strategy
  to other such ridings that exhibit similar responses. As well, the
  party stands to gain important information about what matters most to
  their constituents. Announcing policies about climate change might not
  be as effective as announcing the creation of new jobs in this riding.
  Without knowing this information, however, they might have an
  inefficient media strategy.
\end{itemize}

To sample, the riding was divided into stratums, with the stratums being
the Richmond Hill neighbourhoods. A random sample of the same size was
then collected from each stratum to achieve the goal of at least 3000
respondents. Participants were contacted by phone and through email in
an attempt to maximize response. To protect respondent privacy, no
sensitive or identifiable information such as the name or address of the
respondent was documented.

Results from the survey indicate that the liberals may not be able to
capture this riding again come next election. This begs immediate action
from the party to designate an appropriate action plan. Additionally,
similar such polls should be issued in other swing ridings to identify
which way they lean.

\hypertarget{introduction}{%
\section{Introduction}\label{introduction}}

Richmond hill is the focus of our survey. This riding was chosen on the
basis of it's unstable leanings. It represents an opportunity for the
Liberal party to spend time researching and creating an effective action
plan. Swing ridings are ridings that may change allegiances to parties
between elections, or have hotly contested votes. With it's recent
history and extremely close elections, Richmond hill can be categorized
as swing riding. Readers may be familiar with swing states in the US,
however, the voting system is quite different in Canada. Since provinces
do not vote the same way states do, Canada has swing ridings. They serve
a similar purpose and often become the battlegrounds of political
parties. They represent a keen focus of most political parties, as
elections can come down to the wire, and swing ridings often make the
difference. To provide more context as to why Richmond Hill was chosen,
we must first view its e

\hypertarget{richmond-hill-electory-history}{%
\subsection{Richmond Hill electory
history}\label{richmond-hill-electory-history}}

Richmond Hill has voted mostly Liberal since it's inception as a riding
in 2004. However, the animation below showcases some key points:

\begin{enumerate}
\def\labelenumi{\arabic{enumi}.}
\item
  The Liberals once had a strong hold of the riding, contesting 60\% of
  the votes in 2004. They've been on a sharp decline since, not being
  able to reclaim that status ever since.
\item
  The Conservatives have seen a strong increase in the votes they
  receive. The most recent election in 2019 saw the Liberals beat out
  the conservatives by a slim 112 votes. The Conservatives were able to
  swing the riding once in 2011, but lost out again in 2015.
\item
  The riding is mostly contested by the Liberals and the Conservatives.
  NDP saw an increase in 2011 but soon returned to a previous proportion
  in subsequent years.
\end{enumerate}

Based on projections from Canada338, Richmond Hill would not be red if
an election were held today. This is precisely why The Liberal party
must pay attention to this riding. The GTA is typically a liberal

\begin{verbatim}
## `summarise()` regrouping output by 'years' (override with `.groups` argument)
\end{verbatim}

\includegraphics{2015-07-23-r-rmarkdown_files/figure-latex/unnamed-chunk-2-1.pdf}
\includegraphics{2015-07-23-r-rmarkdown_files/figure-latex/unnamed-chunk-2-2.pdf}
\includegraphics{2015-07-23-r-rmarkdown_files/figure-latex/unnamed-chunk-2-3.pdf}
\includegraphics{2015-07-23-r-rmarkdown_files/figure-latex/unnamed-chunk-2-4.pdf}
\includegraphics{2015-07-23-r-rmarkdown_files/figure-latex/unnamed-chunk-2-5.pdf}
\includegraphics{2015-07-23-r-rmarkdown_files/figure-latex/unnamed-chunk-2-6.pdf}
\includegraphics{2015-07-23-r-rmarkdown_files/figure-latex/unnamed-chunk-2-7.pdf}
\includegraphics{2015-07-23-r-rmarkdown_files/figure-latex/unnamed-chunk-2-8.pdf}
\includegraphics{2015-07-23-r-rmarkdown_files/figure-latex/unnamed-chunk-2-9.pdf}
\includegraphics{2015-07-23-r-rmarkdown_files/figure-latex/unnamed-chunk-2-10.pdf}
\includegraphics{2015-07-23-r-rmarkdown_files/figure-latex/unnamed-chunk-2-11.pdf}
\includegraphics{2015-07-23-r-rmarkdown_files/figure-latex/unnamed-chunk-2-12.pdf}
\includegraphics{2015-07-23-r-rmarkdown_files/figure-latex/unnamed-chunk-2-13.pdf}
\includegraphics{2015-07-23-r-rmarkdown_files/figure-latex/unnamed-chunk-2-14.pdf}
\includegraphics{2015-07-23-r-rmarkdown_files/figure-latex/unnamed-chunk-2-15.pdf}
\includegraphics{2015-07-23-r-rmarkdown_files/figure-latex/unnamed-chunk-2-16.pdf}
\includegraphics{2015-07-23-r-rmarkdown_files/figure-latex/unnamed-chunk-2-17.pdf}
\includegraphics{2015-07-23-r-rmarkdown_files/figure-latex/unnamed-chunk-2-18.pdf}
\includegraphics{2015-07-23-r-rmarkdown_files/figure-latex/unnamed-chunk-2-19.pdf}
\includegraphics{2015-07-23-r-rmarkdown_files/figure-latex/unnamed-chunk-2-20.pdf}
\includegraphics{2015-07-23-r-rmarkdown_files/figure-latex/unnamed-chunk-2-21.pdf}
\includegraphics{2015-07-23-r-rmarkdown_files/figure-latex/unnamed-chunk-2-22.pdf}
\includegraphics{2015-07-23-r-rmarkdown_files/figure-latex/unnamed-chunk-2-23.pdf}
\includegraphics{2015-07-23-r-rmarkdown_files/figure-latex/unnamed-chunk-2-24.pdf}
\includegraphics{2015-07-23-r-rmarkdown_files/figure-latex/unnamed-chunk-2-25.pdf}
\includegraphics{2015-07-23-r-rmarkdown_files/figure-latex/unnamed-chunk-2-26.pdf}
\includegraphics{2015-07-23-r-rmarkdown_files/figure-latex/unnamed-chunk-2-27.pdf}
\includegraphics{2015-07-23-r-rmarkdown_files/figure-latex/unnamed-chunk-2-28.pdf}
\includegraphics{2015-07-23-r-rmarkdown_files/figure-latex/unnamed-chunk-2-29.pdf}
\includegraphics{2015-07-23-r-rmarkdown_files/figure-latex/unnamed-chunk-2-30.pdf}
\includegraphics{2015-07-23-r-rmarkdown_files/figure-latex/unnamed-chunk-2-31.pdf}
\includegraphics{2015-07-23-r-rmarkdown_files/figure-latex/unnamed-chunk-2-32.pdf}
\includegraphics{2015-07-23-r-rmarkdown_files/figure-latex/unnamed-chunk-2-33.pdf}
\includegraphics{2015-07-23-r-rmarkdown_files/figure-latex/unnamed-chunk-2-34.pdf}
\includegraphics{2015-07-23-r-rmarkdown_files/figure-latex/unnamed-chunk-2-35.pdf}
\includegraphics{2015-07-23-r-rmarkdown_files/figure-latex/unnamed-chunk-2-36.pdf}
\includegraphics{2015-07-23-r-rmarkdown_files/figure-latex/unnamed-chunk-2-37.pdf}
\includegraphics{2015-07-23-r-rmarkdown_files/figure-latex/unnamed-chunk-2-38.pdf}
\includegraphics{2015-07-23-r-rmarkdown_files/figure-latex/unnamed-chunk-2-39.pdf}
\includegraphics{2015-07-23-r-rmarkdown_files/figure-latex/unnamed-chunk-2-40.pdf}
\includegraphics{2015-07-23-r-rmarkdown_files/figure-latex/unnamed-chunk-2-41.pdf}
\includegraphics{2015-07-23-r-rmarkdown_files/figure-latex/unnamed-chunk-2-42.pdf}
\includegraphics{2015-07-23-r-rmarkdown_files/figure-latex/unnamed-chunk-2-43.pdf}
\includegraphics{2015-07-23-r-rmarkdown_files/figure-latex/unnamed-chunk-2-44.pdf}
\includegraphics{2015-07-23-r-rmarkdown_files/figure-latex/unnamed-chunk-2-45.pdf}
\includegraphics{2015-07-23-r-rmarkdown_files/figure-latex/unnamed-chunk-2-46.pdf}
\includegraphics{2015-07-23-r-rmarkdown_files/figure-latex/unnamed-chunk-2-47.pdf}
\includegraphics{2015-07-23-r-rmarkdown_files/figure-latex/unnamed-chunk-2-48.pdf}
\includegraphics{2015-07-23-r-rmarkdown_files/figure-latex/unnamed-chunk-2-49.pdf}
\includegraphics{2015-07-23-r-rmarkdown_files/figure-latex/unnamed-chunk-2-50.pdf}
\includegraphics{2015-07-23-r-rmarkdown_files/figure-latex/unnamed-chunk-2-51.pdf}
\includegraphics{2015-07-23-r-rmarkdown_files/figure-latex/unnamed-chunk-2-52.pdf}
\includegraphics{2015-07-23-r-rmarkdown_files/figure-latex/unnamed-chunk-2-53.pdf}
\includegraphics{2015-07-23-r-rmarkdown_files/figure-latex/unnamed-chunk-2-54.pdf}
\includegraphics{2015-07-23-r-rmarkdown_files/figure-latex/unnamed-chunk-2-55.pdf}
\includegraphics{2015-07-23-r-rmarkdown_files/figure-latex/unnamed-chunk-2-56.pdf}
\includegraphics{2015-07-23-r-rmarkdown_files/figure-latex/unnamed-chunk-2-57.pdf}
\includegraphics{2015-07-23-r-rmarkdown_files/figure-latex/unnamed-chunk-2-58.pdf}
\includegraphics{2015-07-23-r-rmarkdown_files/figure-latex/unnamed-chunk-2-59.pdf}
\includegraphics{2015-07-23-r-rmarkdown_files/figure-latex/unnamed-chunk-2-60.pdf}
\includegraphics{2015-07-23-r-rmarkdown_files/figure-latex/unnamed-chunk-2-61.pdf}
\includegraphics{2015-07-23-r-rmarkdown_files/figure-latex/unnamed-chunk-2-62.pdf}
\includegraphics{2015-07-23-r-rmarkdown_files/figure-latex/unnamed-chunk-2-63.pdf}
\includegraphics{2015-07-23-r-rmarkdown_files/figure-latex/unnamed-chunk-2-64.pdf}
\includegraphics{2015-07-23-r-rmarkdown_files/figure-latex/unnamed-chunk-2-65.pdf}
\includegraphics{2015-07-23-r-rmarkdown_files/figure-latex/unnamed-chunk-2-66.pdf}
\includegraphics{2015-07-23-r-rmarkdown_files/figure-latex/unnamed-chunk-2-67.pdf}
\includegraphics{2015-07-23-r-rmarkdown_files/figure-latex/unnamed-chunk-2-68.pdf}
\includegraphics{2015-07-23-r-rmarkdown_files/figure-latex/unnamed-chunk-2-69.pdf}
\includegraphics{2015-07-23-r-rmarkdown_files/figure-latex/unnamed-chunk-2-70.pdf}
\includegraphics{2015-07-23-r-rmarkdown_files/figure-latex/unnamed-chunk-2-71.pdf}
\includegraphics{2015-07-23-r-rmarkdown_files/figure-latex/unnamed-chunk-2-72.pdf}
\includegraphics{2015-07-23-r-rmarkdown_files/figure-latex/unnamed-chunk-2-73.pdf}
\includegraphics{2015-07-23-r-rmarkdown_files/figure-latex/unnamed-chunk-2-74.pdf}
\includegraphics{2015-07-23-r-rmarkdown_files/figure-latex/unnamed-chunk-2-75.pdf}
\includegraphics{2015-07-23-r-rmarkdown_files/figure-latex/unnamed-chunk-2-76.pdf}
\includegraphics{2015-07-23-r-rmarkdown_files/figure-latex/unnamed-chunk-2-77.pdf}
\includegraphics{2015-07-23-r-rmarkdown_files/figure-latex/unnamed-chunk-2-78.pdf}
\includegraphics{2015-07-23-r-rmarkdown_files/figure-latex/unnamed-chunk-2-79.pdf}
\includegraphics{2015-07-23-r-rmarkdown_files/figure-latex/unnamed-chunk-2-80.pdf}
\includegraphics{2015-07-23-r-rmarkdown_files/figure-latex/unnamed-chunk-2-81.pdf}
\includegraphics{2015-07-23-r-rmarkdown_files/figure-latex/unnamed-chunk-2-82.pdf}
\includegraphics{2015-07-23-r-rmarkdown_files/figure-latex/unnamed-chunk-2-83.pdf}
\includegraphics{2015-07-23-r-rmarkdown_files/figure-latex/unnamed-chunk-2-84.pdf}
\includegraphics{2015-07-23-r-rmarkdown_files/figure-latex/unnamed-chunk-2-85.pdf}
\includegraphics{2015-07-23-r-rmarkdown_files/figure-latex/unnamed-chunk-2-86.pdf}
\includegraphics{2015-07-23-r-rmarkdown_files/figure-latex/unnamed-chunk-2-87.pdf}
\includegraphics{2015-07-23-r-rmarkdown_files/figure-latex/unnamed-chunk-2-88.pdf}
\includegraphics{2015-07-23-r-rmarkdown_files/figure-latex/unnamed-chunk-2-89.pdf}
\includegraphics{2015-07-23-r-rmarkdown_files/figure-latex/unnamed-chunk-2-90.pdf}
\includegraphics{2015-07-23-r-rmarkdown_files/figure-latex/unnamed-chunk-2-91.pdf}
\includegraphics{2015-07-23-r-rmarkdown_files/figure-latex/unnamed-chunk-2-92.pdf}
\includegraphics{2015-07-23-r-rmarkdown_files/figure-latex/unnamed-chunk-2-93.pdf}
\includegraphics{2015-07-23-r-rmarkdown_files/figure-latex/unnamed-chunk-2-94.pdf}
\includegraphics{2015-07-23-r-rmarkdown_files/figure-latex/unnamed-chunk-2-95.pdf}
\includegraphics{2015-07-23-r-rmarkdown_files/figure-latex/unnamed-chunk-2-96.pdf}
\includegraphics{2015-07-23-r-rmarkdown_files/figure-latex/unnamed-chunk-2-97.pdf}
\includegraphics{2015-07-23-r-rmarkdown_files/figure-latex/unnamed-chunk-2-98.pdf}
\includegraphics{2015-07-23-r-rmarkdown_files/figure-latex/unnamed-chunk-2-99.pdf}
\includegraphics{2015-07-23-r-rmarkdown_files/figure-latex/unnamed-chunk-2-100.pdf}

\hypertarget{survey-methodology}{%
\section{Survey methodology}\label{survey-methodology}}

The survey:
\url{https://www.surveymonkey.com/r/D8KBRNN?fbclid=IwAR1iL8B_kYBasO4GNID5cdS44hBCHFtMgJsRToi31fUFA68Ag2d1YfnTYjQ}

The target population of the survey is all the residents of Richmond
hill Riding that are eligible to vote. The frame is all the people of
the riding whose email-address or phone number are available to us. The
sample is all the people among the frame population who are willing to
take our surveys. To sample, we used stratified sampling, separating the
riding into stratums and randomly collecting samples without replacement
from each stratum. Each stratum represents a neighbourhood in the riding
from the following: - Elgin Mills - Bayview North - Bayview South -
North Richvale - Hillsview - Bayview Hill - South Richvale - Langstaff -
Doncrest

Therefore, each stratum has the same proportion in the sample as it does
in the target population. Also, any resident can be selected at most
once because everyone has only one vote. The theorem of SRSWOR suggests
that the design-based variance of \[\bar{y}\] is given by:
\[V(\bar{y}) = (1 - (n/N)*(\sigma^2_y/n)\]. Intuitively, this means that
we can expect for the variance of each sampled stratum to be very small
when the sample size is large relative to the population. The theorem
also tells us that the sample mean \[\bar{y}\] is a design-unbiased
estimator for the true population mean: \[E(\bar{y}) = \mu_y\]. We will
use phone calls and email to reach our desired respondents, and each
survey is estimated to take about 5 minutes.

To reduce non-response, we have tried to make the survey more
user-friendly, attractive(clean interface), non-intrusive(doesn't
collect any personal information), and time-conscious. If the
non-responses are few and are randomly spread in the target population,
then they can be neglected in the analysis. Non-response in certain
questions can lead to different numbers of observations in the same
variable, which might make it harder to construct a model. Non-response
of some people(empty questionnaires) will make our sample less
reflective if they are from some specific groups, which could lead to
bias towards a particular group. In this case, we need to adjust the
weights in our sample to have a sample reflective of the population we
wish to observe.

Finally to protect respondent privacy, the entire survey will be
anonymous and includes no information which indicates respondents'
identity. Additionally, we're greatly limiting the types of information
we're collecting that could be qualified as Personally Identifiable
Information(PII). Also, in terms of some relatively sensitive
information such as age and household income, respondents just need to
select a range instead of the exact number to further ensure
confidentiality.

\hypertarget{results}{%
\section{Results}\label{results}}

\hypertarget{how-have-respondents-historically-voted-and-current-attitudes}{%
\subsection{How have respondents historically voted and current
attitudes?}\label{how-have-respondents-historically-voted-and-current-attitudes}}

Looking at Figure 2 and Figure 3, the results are quite close. MP Majid
Jowhari was the incumbent in the 2019 elections and barely won. The gaps
between the conservatives and liberals has become even more slim.

\begin{Shaded}
\begin{Highlighting}[]
\CommentTok{#Simulation of Q:  Whom did you vote for in the last elections?}

\NormalTok{survey_results }\OperatorTok\StringTok{ }
\KeywordTok{ggplot}\NormalTok{(}\KeywordTok{aes}\NormalTok{(}\DataTypeTok{x =}\NormalTok{ election_results_}\DecValTok{2019}\NormalTok{, }\DataTypeTok{fill =}\NormalTok{ election_results_}\DecValTok{2019}\NormalTok{)) }\OperatorTok{+}\StringTok{ }
\KeywordTok{geom_bar}\NormalTok{() }\OperatorTok{+}\StringTok{ }
\KeywordTok{scale_fill_manual}\NormalTok{(}\DataTypeTok{values =} \KeywordTok{c}\NormalTok{(}\StringTok{"blue"}\NormalTok{, }\StringTok{"green"}\NormalTok{, }\StringTok{"red"}\NormalTok{, }\StringTok{"orange"}\NormalTok{, }\StringTok{"brown"}\NormalTok{, }\StringTok{"pink"}\NormalTok{)) }\OperatorTok{+}
\KeywordTok{labs}\NormalTok{(}\DataTypeTok{title =} \StringTok{"Figure 2: Which party did you vote for in the 2019 federal elections?"}\NormalTok{,}
     \DataTypeTok{subtitle =} \StringTok{"The People's party and Rhinoceros party were new additions to this election."}\NormalTok{,}
     \DataTypeTok{caption =} \StringTok{"Source: Simulated data."}\NormalTok{,}
     \DataTypeTok{x =} \StringTok{"Federal Party"}\NormalTok{,}
     \DataTypeTok{y =} \StringTok{"Number of votes"}\NormalTok{,}
     \DataTypeTok{fill =} \StringTok{"Party"}\NormalTok{,}
     \DataTypeTok{tag =} \StringTok{"[2]"}\NormalTok{)}
\end{Highlighting}
\end{Shaded}

\includegraphics{2015-07-23-r-rmarkdown_files/figure-latex/unnamed-chunk-4-1.pdf}

\begin{Shaded}
\begin{Highlighting}[]
\CommentTok{#Simulation of Q:  Whom did you vote for in the 2015 elections?}
\NormalTok{survey_results }\OperatorTok
\KeywordTok{ggplot}\NormalTok{(}\KeywordTok{aes}\NormalTok{(}\DataTypeTok{x =}\NormalTok{ election_results_}\DecValTok{2015}\NormalTok{, }\DataTypeTok{fill =}\NormalTok{ election_results_}\DecValTok{2015}\NormalTok{)) }\OperatorTok{+}\StringTok{ }
\KeywordTok{geom_bar}\NormalTok{() }\OperatorTok{+}\StringTok{ }
\KeywordTok{scale_fill_manual}\NormalTok{(}\DataTypeTok{values =} \KeywordTok{c}\NormalTok{(}\StringTok{"blue"}\NormalTok{, }\StringTok{"green"}\NormalTok{, }\StringTok{"red"}\NormalTok{, }\StringTok{"orange"}\NormalTok{, }\StringTok{"brown"}\NormalTok{, }\StringTok{"pink"}\NormalTok{)) }\OperatorTok{+}
\KeywordTok{labs}\NormalTok{(}\DataTypeTok{title =} \StringTok{"Figure 3: Which party did you vote for in the 2015 federal elections?"}\NormalTok{,}
     \DataTypeTok{caption =} \StringTok{"Source: Simulated data."}\NormalTok{,}
     \DataTypeTok{x =} \StringTok{"Federal Party"}\NormalTok{,}
     \DataTypeTok{y =} \StringTok{"Number of votes"}\NormalTok{,}
     \DataTypeTok{fill =} \StringTok{"Party"}\NormalTok{,}
     \DataTypeTok{tag =} \StringTok{"[3]"}\NormalTok{)}
\end{Highlighting}
\end{Shaded}

\includegraphics{2015-07-23-r-rmarkdown_files/figure-latex/unnamed-chunk-5-1.pdf}
In the 2015 elections, the Liberals were able to swing the riding back
from the conservatives, albeit a close result.

\begin{Shaded}
\begin{Highlighting}[]
\CommentTok{#Q: Whom did you vote for in the provincial elections?}

\NormalTok{survey_results }\OperatorTok\StringTok{ }
\StringTok{  }\KeywordTok{ggplot}\NormalTok{(}\KeywordTok{aes}\NormalTok{(}\DataTypeTok{x =}\NormalTok{ elections_results_provincial_}\DecValTok{2020}\NormalTok{, }\DataTypeTok{fill =}\NormalTok{ elections_results_provincial_}\DecValTok{2020}\NormalTok{)) }\OperatorTok{+}\StringTok{ }
\StringTok{  }\KeywordTok{geom_bar}\NormalTok{() }\OperatorTok{+}\StringTok{ }
\StringTok{  }\KeywordTok{coord_flip}\NormalTok{() }\OperatorTok{+}
\StringTok{  }\KeywordTok{scale_fill_manual}\NormalTok{(}\DataTypeTok{values =} \KeywordTok{c}\NormalTok{(}\StringTok{"green"}\NormalTok{, }\StringTok{"red"}\NormalTok{, }\StringTok{"brown"}\NormalTok{, }\StringTok{"orange"}\NormalTok{, }\StringTok{"blue"}\NormalTok{)) }\OperatorTok{+}\StringTok{ }
\StringTok{  }\KeywordTok{labs}\NormalTok{(}\DataTypeTok{title =} \StringTok{"Figure 4: Which party did you vote for in the 2018 Provincial Elections?"}\NormalTok{,}
       \DataTypeTok{caption =} \StringTok{"Source: Simulated data."}\NormalTok{,}
       \DataTypeTok{x =} \StringTok{"Type of spending"}\NormalTok{,}
       \DataTypeTok{y =} \StringTok{"Number of people"}\NormalTok{,}
       \DataTypeTok{fill =} \StringTok{"Provincial Party Legend"}\NormalTok{,}
       \DataTypeTok{tag =} \StringTok{"[4]"}\NormalTok{)}
\end{Highlighting}
\end{Shaded}

\includegraphics{2015-07-23-r-rmarkdown_files/figure-latex/unnamed-chunk-6-1.pdf}
Figure 4 showcases the crushing victory of the Progressive Conservative
Party of Ontario. Federal election results may mirror that of the
provincial elections. This seems to be confirmed by figure 5.

\begin{Shaded}
\begin{Highlighting}[]
\CommentTok{#Simulation of Q:  If you could vote tomorrow, which of the parties would you vote for?}

\NormalTok{survey_results }\OperatorTok\StringTok{ }
\KeywordTok{ggplot}\NormalTok{(}\KeywordTok{aes}\NormalTok{(}\DataTypeTok{x =}\NormalTok{ election_results_}\DecValTok{2020}\NormalTok{, }\DataTypeTok{fill =}\NormalTok{ election_results_}\DecValTok{2020}\NormalTok{)) }\OperatorTok{+}\StringTok{ }
\KeywordTok{geom_bar}\NormalTok{() }\OperatorTok{+}\StringTok{ }
\KeywordTok{scale_fill_manual}\NormalTok{(}\DataTypeTok{values =} \KeywordTok{c}\NormalTok{(}\StringTok{"blue"}\NormalTok{, }\StringTok{"green"}\NormalTok{, }\StringTok{"red"}\NormalTok{, }\StringTok{"orange"}\NormalTok{, }\StringTok{"brown"}\NormalTok{, }\StringTok{"pink"}\NormalTok{)) }\OperatorTok{+}
\KeywordTok{labs}\NormalTok{(}\DataTypeTok{title =} \StringTok{"Figure 5: If you could vote tomorrow, which of the parties would you vote for?"}\NormalTok{,}
     \DataTypeTok{caption =} \StringTok{"Source: Simulated data."}\NormalTok{,}
     \DataTypeTok{x =} \StringTok{"Federal Party"}\NormalTok{,}
     \DataTypeTok{y =} \StringTok{"Number of votes"}\NormalTok{,}
     \DataTypeTok{fill =} \StringTok{"Party"}\NormalTok{,}
     \DataTypeTok{tag =} \StringTok{"[5]"}\NormalTok{)}
\end{Highlighting}
\end{Shaded}

\includegraphics{2015-07-23-r-rmarkdown_files/figure-latex/unnamed-chunk-7-1.pdf}
Figure 5 is not a good result for the Liberal party. It shows that most
respondents would now vote Conservative. The riding may now be swinging
towards the conservatives.

\begin{Shaded}
\begin{Highlighting}[]
\CommentTok{#Simulation of Q:  Overall, how would you rate the performance of the current federal government party(1-10)?}
\NormalTok{survey_results }\OperatorTok\StringTok{ }
\StringTok{  }\KeywordTok{ggplot}\NormalTok{(}\KeywordTok{aes}\NormalTok{(}\DataTypeTok{x =}\NormalTok{ current_government_score)) }\OperatorTok{+}\StringTok{ }
\StringTok{  }\KeywordTok{geom_histogram}\NormalTok{() }\OperatorTok{+}\StringTok{  }
\StringTok{  }\KeywordTok{labs}\NormalTok{(}\DataTypeTok{title =} \StringTok{"Figure 6: Rate the performance of the current federal government party?"}\NormalTok{,}
       \DataTypeTok{subtitle =} \StringTok{"The rating was on a scale of 1-10"}\NormalTok{,}
       \DataTypeTok{caption =} \StringTok{"Source: Simulated data."}\NormalTok{,}
       \DataTypeTok{x =} \StringTok{"Rating"}\NormalTok{,}
       \DataTypeTok{y =} \StringTok{"Number of responses"}\NormalTok{,}
       \DataTypeTok{tag =} \StringTok{"[6]"}\NormalTok{)}
\CommentTok{## `stat_bin()` using `bins = 30`. Pick better value with `binwidth`.}
\end{Highlighting}
\end{Shaded}

\includegraphics{2015-07-23-r-rmarkdown_files/figure-latex/unnamed-chunk-8-1.pdf}
Figure 6 indicates a poor approval rating from the respondents. With an
average score of 5, residents don't seem to be pleased with the current
performance of the party

\hypertarget{richmond-hill-key-demographic-variables}{%
\subsection{Richmond Hill key demographic
variables}\label{richmond-hill-key-demographic-variables}}

\begin{Shaded}
\begin{Highlighting}[]
\CommentTok{#Q: What range does your household income fall into?}


\NormalTok{survey_results }\OperatorTok\StringTok{ }
\KeywordTok{ggplot}\NormalTok{(}\KeywordTok{aes}\NormalTok{(}\DataTypeTok{x =}\NormalTok{ income_range_responses, }\DataTypeTok{fill =}\NormalTok{ income_range_responses)) }\OperatorTok{+}
\KeywordTok{geom_bar}\NormalTok{() }\OperatorTok{+}\StringTok{ }\KeywordTok{coord_flip}\NormalTok{() }\OperatorTok{+}
\KeywordTok{labs}\NormalTok{(}\DataTypeTok{title =} \StringTok{"Figure 7: Income range of residents of Richmond Hill"}\NormalTok{,}
     \DataTypeTok{caption =} \StringTok{"Source: Simulated data."}\NormalTok{,}
     \DataTypeTok{x =} \StringTok{"Income Range"}\NormalTok{,}
     \DataTypeTok{y =} \StringTok{"Number of people"}\NormalTok{,}
     \DataTypeTok{fill =} \StringTok{"Income range"}\NormalTok{,}
     \DataTypeTok{tag =} \StringTok{"[7]"}\NormalTok{) }\OperatorTok{+}\StringTok{ }\KeywordTok{theme}\NormalTok{(}\DataTypeTok{legend.position =} \StringTok{"none"}\NormalTok{)}
\end{Highlighting}
\end{Shaded}

\includegraphics{2015-07-23-r-rmarkdown_files/figure-latex/unnamed-chunk-9-1.pdf}
In figure 7, it is clear that Richmond hill has a high proportion of
residents making more than \$100,000.

\begin{Shaded}
\begin{Highlighting}[]
\CommentTok{#Q: What is your age group?}


\NormalTok{survey_results }\OperatorTok\StringTok{ }
\KeywordTok{ggplot}\NormalTok{(}\KeywordTok{aes}\NormalTok{(}\DataTypeTok{x =}\NormalTok{ age_range_responses, }\DataTypeTok{fill =}\NormalTok{ age_range_responses)) }\OperatorTok{+}\StringTok{ }
\KeywordTok{geom_bar}\NormalTok{()  }\OperatorTok{+}
\KeywordTok{labs}\NormalTok{(}\DataTypeTok{title =} \StringTok{"Figure 8: Distribution of age in Richmond Hill"}\NormalTok{,}
     \DataTypeTok{caption =} \StringTok{"Source: Simulated data."}\NormalTok{,}
     \DataTypeTok{x =} \StringTok{"Age group"}\NormalTok{,}
     \DataTypeTok{y =} \StringTok{"Number of people"}\NormalTok{,}
     \DataTypeTok{fill =} \StringTok{"Age group brackets"}\NormalTok{,}
     \DataTypeTok{tag =} \StringTok{"[8]"}\NormalTok{) }\OperatorTok{+}\StringTok{ }\KeywordTok{theme}\NormalTok{(}\DataTypeTok{legend.position =} \StringTok{"none"}\NormalTok{)}
\end{Highlighting}
\end{Shaded}

\includegraphics{2015-07-23-r-rmarkdown_files/figure-latex/unnamed-chunk-10-1.pdf}

Richmond hill has a fairly uniform spread for age. It has a strong peak
in 45-54 age group, signalling an aging population.

\begin{Shaded}
\begin{Highlighting}[]
\CommentTok{#Q Which of these issues would you consider to be of most importance ahead of the next election}
\NormalTok{survey_results }\OperatorTok\StringTok{ }
\KeywordTok{ggplot}\NormalTok{(}\KeywordTok{aes}\NormalTok{(}\DataTypeTok{x =}\NormalTok{ most_important_issues, }\DataTypeTok{fill =}\NormalTok{ most_important_issues)) }\OperatorTok{+}\StringTok{ }
\KeywordTok{geom_bar}\NormalTok{()  }\OperatorTok{+}\StringTok{ }\KeywordTok{coord_flip}\NormalTok{() }\OperatorTok{+}\StringTok{ }
\KeywordTok{labs}\NormalTok{(}\DataTypeTok{title =} \StringTok{"Figure 9: Most important issues"}\NormalTok{,}
     \DataTypeTok{caption =} \StringTok{"Source: Simulated data."}\NormalTok{,}
     \DataTypeTok{x =} \StringTok{"Issue"}\NormalTok{,}
     \DataTypeTok{y =} \StringTok{"Number of people"}\NormalTok{,}
     \DataTypeTok{fill =} \StringTok{"Issue"}\NormalTok{,}
     \DataTypeTok{tag =} \StringTok{"[9]"}\NormalTok{) }\OperatorTok{+}\StringTok{ }\KeywordTok{theme}\NormalTok{(}\DataTypeTok{legend.position =} \StringTok{"none"}\NormalTok{)}
\end{Highlighting}
\end{Shaded}

\includegraphics{2015-07-23-r-rmarkdown_files/figure-latex/unnamed-chunk-11-1.pdf}

\begin{Shaded}
\begin{Highlighting}[]
\CommentTok{#Q Which of these issues would you consider to be of least importance ahead of the next election}

\NormalTok{survey_results }\OperatorTok\StringTok{ }
\KeywordTok{ggplot}\NormalTok{(}\KeywordTok{aes}\NormalTok{(}\DataTypeTok{x =}\NormalTok{ least_important_issues, }\DataTypeTok{fill =}\NormalTok{ least_important_issues)) }\OperatorTok{+}\StringTok{ }
\KeywordTok{geom_bar}\NormalTok{()  }\OperatorTok{+}\StringTok{ }\KeywordTok{coord_flip}\NormalTok{() }\OperatorTok{+}
\KeywordTok{labs}\NormalTok{(}\DataTypeTok{title =} \StringTok{"Figure 10: Least important issues"}\NormalTok{,}
     \DataTypeTok{caption =} \StringTok{"Source: Simulated data."}\NormalTok{,}
     \DataTypeTok{x =} \StringTok{"Issue"}\NormalTok{,}
     \DataTypeTok{y =} \StringTok{"Number of people"}\NormalTok{,}
     \DataTypeTok{fill =} \StringTok{"Issue"}\NormalTok{,}
     \DataTypeTok{tag =} \StringTok{"[10]"}\NormalTok{) }\OperatorTok{+}\StringTok{ }\KeywordTok{theme}\NormalTok{(}\DataTypeTok{legend.position =} \StringTok{"none"}\NormalTok{)}
\end{Highlighting}
\end{Shaded}

\includegraphics{2015-07-23-r-rmarkdown_files/figure-latex/unnamed-chunk-12-1.pdf}
From figures 9 and 10, we gather that the residents of Richmond hill
find some issues much more important than others. Primarily, they are
concerned with unemployment, immigration, and affordable housing.

\hypertarget{discussions}{%
\section{Discussions}\label{discussions}}

The results of the survey can be distilled into a few key points:

\begin{itemize}
\tightlist
\item
  The Liberal party has had close calls in their 2 recent elections.
  Even in 2015, when the Liberals were able to reclaim GTA from the
  Conservatives.
\end{itemize}

\includegraphics{img/2011_riding_map.png}
\includegraphics{img/2015_riding_map.png}

\hypertarget{weaknesses-and-biases}{%
\section{Weaknesses and Biases}\label{weaknesses-and-biases}}

\begin{itemize}
\item
  We consider all the eligible voters of the riding as our target
  population and select samples from them. However, in the 2015
  election, 38.6\% of the total eligible voters did not vote. If some
  specific groups respond to the survey but do not vote in the end, our
  estimation of the election will very likely be distorted.
\item
  Non-response is another problem. If a considerable number of people
  with similar characteristics do not complete the survey (including
  those who finish it partially), we are unable to get information from
  them, which will make our model incomplete and potentially misleading.
\item
  It is based on a few assumptions. One of the assumptions is static
  performance of the party. Any number of things could happen before the
  next election that may eventually swing the riding back into the
  liberals. Another assumption is static demographics of the riding. One
  variable that was not measured was the ethnic make up of the
  constituents. Certain demographic shifts may have the effect of
  shifting the votes.
\end{itemize}

\hypertarget{references}{%
\section{References}\label{references}}

\begin{itemize}
\item
  CBC. (2015). Federal election results 2015. Retrieved from the CBC
  website: \url{https://www.cbc.ca/news2/interactives/results-2015/}
\item
  Elections Ontario. (2018). Valid Votes Cast for Each Candidate-2018
  General election. Retrieved from the Elections Ontario Website:
  \url{https://results.elections.on.ca/en/publications}
\item
  Elections Canada (2004). Official Voting results. Retrieved from the
  Elections Canada Website:
  \url{https://www.elections.ca/scripts/OVR2004/default.html}
\item
  Elections Canada (2006). Thirty-ninth General Election 2006: Official
  Voting Results. Retrieved from the Elections Canada Website:
  \url{https://www.elections.ca/content.aspx?section=res\&dir=rep/off/39gedata\&document=summary\&lang=e}
\item
  Elections Canada (2008). OFFICIAL VOTING RESULTS FORTIETH GENERAL
  ELECTION 2008. Retrieved from the Elections Canada Website:
  \url{https://www.elections.ca/scripts/OVR2008/default.html}
\item
  Elections Canada (2011). OFFICIAL VOTING RESULTS FORTY-FIRST GENERAL
  ELECTION 2011. Retrieved from the Elections Canada Website:
  \url{https://www.elections.ca/scripts/ovr2011/default.html}
\item
  Elections Canada (2015). FORTY-SECOND GENERAL ELECTION 2015 ---
  Poll-by-poll results. Retrieved from the Elections Canada Website:
  \url{https://www.elections.ca/res/rep/off/ovr2015app/41/9878e.html}
\item
  Elections Canada (2019). FORTY-THIRD GENERAL ELECTION 2019 ---
  Poll-by-poll results. Retrieved from the Elections Canada
  Website:\url{https://www.elections.ca/res/rep/off/ovr2019app/51/11129e.html}
\item
  JJ Allaire, Jeffrey Horner, Yihui Xie, Vicent Marti and Natacha Porte
  (2019). markdown: Render Markdown with the C Library `Sundown'. R
  package version 1.1. \url{https://CRAN.R-project.org/package=markdown}
\item
  Jeroen Ooms (2018). gifski: Highest Quality GIF Encoder. R package
  version 0.8.6. \url{https://CRAN.R-project.org/package=gifski}
\item
  R Core Team (2020). R: A language and environment for statistical
  computing. R, Foundation for Statistical Computing, Vienna, Austria.
  URL, \url{https://www.R-project.org/}.
\item
  Simon Urbanek (2013). png: Read and write PNG images. R package
  version 0.1-7. \url{https://CRAN.R-project.org/package=png}
\item
  Statistics Canada. 2017. Richmond Hill, T {[}Census subdivision{]},
  Ontario and Ontario {[}Province{]} (table). Census Profile. 2016
  Census. Statistics Canada Catalogue no. 98-316-X2016001. Ottawa.
  Released November 29, 2017.
  \url{https://www12.statcan.gc.ca/census-recensement/2016/dp-pd/prof/index.cfm?Lang=E}
  (accessed October 6, 2020)
\item
  Thomas Lin Pedersen and David Robinson (2020). gganimate: A Grammar of
  Animated Graphics. R package version 1.0.6.
  \url{https://CRAN.R-project.org/package=gganimate}
\item
  Thomas Lin Pedersen (2020). patchwork: The Composer of Plots. R
  package version 1.0.1.
  \url{https://CRAN.R-project.org/package=patchwork}
\item
  Wickham et al., (2019). Welcome to the tidyverse. Journal of Open
  Source Software, 4(43), 1686,
  \url{https://doi.org/10.21105/joss.01686}
\item
  Yihui Xie (2020). blogdown: Create Blogs and Websites with R Markdown.
  R package version 0.20.
\end{itemize}

\hypertarget{appendix}{%
\section{Appendix}\label{appendix}}

\includegraphics{img/income_range.png}
\includegraphics{img/age_range.png}

\includegraphics{img/2019_result.png}

\includegraphics{img/important_issues.png}

\begin{Shaded}
\begin{Highlighting}[]
\NormalTok{knitr}\OperatorTok{::}\KeywordTok{include_graphics}\NormalTok{(}\StringTok{'img/age_range.png'}\NormalTok{)}
\end{Highlighting}
\end{Shaded}

\includegraphics[width=9.83in]{img/age_range}

\begin{Shaded}
\begin{Highlighting}[]
\NormalTok{knitr}\OperatorTok{::}\KeywordTok{include_graphics}\NormalTok{(}\StringTok{'img/2019_result.png'}\NormalTok{)}
\end{Highlighting}
\end{Shaded}

\includegraphics[width=10.43in]{img/2019_result}

\end{document}
